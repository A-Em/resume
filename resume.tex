% !TEX program = xelatex

\documentclass{resume}
%\usepackage{zh_CN-Adobefonts_external} % Simplified Chinese Support using external fonts (./fonts/zh_CN-Adobe/)
%\usepackage{zh_CN-Adobefonts_internal} % Simplified Chinese Support using system fonts

\begin{document}
\pagenumbering{gobble} % suppress displaying page number

\name{Yinghao Yuan}

\basicInfo{
  \email{2021202210083@whu.edu.cn} \textperiodcentered\ 
  \phone{(+86) 152-2101-5199}
  }

\section{\faGraduationCap\ Education}
\datedsubsection{\textbf{Wuhan University (WHU)}, Wuhan, China}{2021 -- 2024}
\textit{Master student} in Cyberspace Security (CS)
\datedsubsection{\textbf{China University of Geosciences}, Wuhan, China}{2017 -- 2021}
\textit{B.S.} in Information Management and Information System (IMIS)

\section{\faUsers\ Practical Experience}

\datedsubsection{\textbf{Hubei Provincial Key Research Program, Research on Key Technologies of Domestic Autonomous and Controllable Security for Internet of Things}}{Sep. 2022 -- May. 2024}
\begin{itemize}
  \item Aiming at the current heterogeneous and diverse IoT devices, there are performance, cost, power consumption and other constraints on the device being applied in practice, I worked with other members of the lab on the research on the security needs of IoT devices, and then from the point of view of the attackers, designers, operation and maintenance managers, we developed a management platform for the secure access of IoT devices. This research breaks through key technologies such as endogenous trustworthiness of AI-integrated IoT, intelligent situational awareness and defence for IoT traffic.
\end{itemize}

\datedsubsection{\textbf{Beijing Electric Power Research Institute Program, Research on Energy Internet Security Interconnection Mechanism}}{Sep. 2022 -- May. 2024}
\begin{itemize}
  \item As the main project member, my main work is to design and implement a set of comprehensive trustworthiness measures and security protection programs to isolate various malicious operations and ensure the system is always in a trustworthy state in response to the problems of difficult-to-trust energy Internet terminals and the failure to guarantee the integrity of massive terminals, etc. My project thesis, "Security Protection Method of Energy Internet with Android," as well as the related patents, have been published.
\end{itemize}

\datedsubsection{\textbf{Alibaba's project, Research on Trusted Computing Program for Open Platforms}}{Sep. 2022 -- May. 2024}
\begin{itemize}
  \item Facing the security problems of container scenarios, based on trusted computing, I designed a set of solutions together with my research team: utilizing dynamic metrics to measure the integrity of runtime program code segments in memory; adopting trusted network connections for access authentication during container interactions; formulating a trust level classification scheme for hosts based on the existing research and designing a real-time response to recover the system's base file scheme.
\end{itemize}

% Reference Test
%\datedsubsection{\textbf{Paper Title\cite{zaharia2012resilient}}}{May. 2015}
%An xxx optimized for xxx\cite{verma2015large}
%\begin{itemize}
%  \item main contribution
%\end{itemize}

\section{\faInfo\ Research Experience}

\datedsubsection{\textbf{Research on the construction method of trusted air and space information transmission link}}{Sep. 2021 -- Dec. 2021}
\begin{itemize}
  \item As the main project member, I have researched on the security of airspace networks from the levels of terminal environment security, network identity security, and data transmission security and put forward a solution. This research adopts the trusted startup method based on the idea of trusted computing to ensure the security of the terminal itself when it starts up, and at the same time ensure the security of the usage environment; introduces the trusted network connection technology to authenticate the identity of the access terminal, and at the same time uses the secure encryption protocol, so as to solve the problem of data transmission security from end to end. In addition, I have produced one SCI paper(Zone One) "Trust Assessment Under the Integrated Air-Space-Ground Network Environment", of which I am the second author.
\end{itemize}

\datedsubsection{\textbf{Privacy Threats and Countermeasures in the Internet of Vehicles}}{Dec. 2022 -- Jun. 2023}
\begin{itemize}
  \item In this study, the current privacy threats to vehicles and occupants in Telematics are sorted out, and the corresponding countermeasures are categorized to guide the subsequent research proposals. In addition, this study also devises an idealized domain-wide privacy protection scheme. The article is currently under submission.
\end{itemize}

\datedsubsection{\textbf{\datedsubsection{\textbfA Privacy-Preserving Framework for Object Detection based on Fluid MPC}}{Jun. 2023 -- Jan. 2024}
\begin{itemize}
  \item Edge devices with image classification needs have difficulty processing private data by themselves and often need to hand over private images to a centralized server for appropriate processing. Now, most of the so-called "lightweight" privacy-preserving techniques require a large amount of computational and communication resources of the computational participants and require the participants to stay online all the time, which is further compounded by the fact that current technological solutions consider the participants to be "dishonest" only. The fact that all current technology schemes consider participants with only "dishonest" characteristics makes it even more difficult to apply in practice. Based on the concept of Fluid MPC, This study proposes a privacy-preserving technique for image processing using hash functions, the SPDZ protocol, and homomorphic encryption constructs, which is capable of tracking every step of a participant's attributes and securely completing the data transmission, allowing computational participants to come and go freely. Currently in the formative article.
\end{itemize}

\section{\faHeartO\ Honors and Awards}
\begin{itemize}[parsep=0.5ex]
  \item Outstanding Graduates, 2024
  \item Outstanding Graduate Student, 2022&2023
  \item Outstanding Student Cadre of Wuhan University, 2023
  \item Academic First Class Scholarship of Wuhan University, 2022
  \item Social Practice Advanced Individual of Wuhan University, 2022
\end{itemize}

\section{\faCogs\ Skills}
\begin{itemize}[parsep=0.5ex]
  \item Programming Languages: C, C++, Rust == Python > Java
\end{itemize}

%% Reference
%\newpage
%\bibliographystyle{IEEETran}
%\bibliography{mycite}
\end{document}
